% !Mode:: "TeX:UTF-8"
% !TEX program = xelatex

% @Author:             old jia
% @Email:              jiaminxin@outlook.com
% @Date:               2021-11-01 17:10:06
% @Last Modified by:   Administrator
% @Last Modified time: 2021-11-01 18:19:58


\documentclass[12pt,a4paper,hyperref]{ctexrep}
	\usepackage{amsmath,amsfonts,amscd,amsthm,dsfont,amssymb,extarrows}
	\usepackage{color}                                                               
	\usepackage{ccmap}
	\usepackage{listings}
	\title{自动推导离散谱问题中V的Maple实现}
	\author{JMx}
	\date{\today}

\begin{document}
	\maketitle
	
这里我们记录一下使用Maple自动推导离散谱问题的时间部分的V的形式。

我们以$4$阶谱问题为例。 给定空间谱问题$U(n, \lambda)=(U_{ij})_{4\times 4}$ (这里我们限定$U$中的同一位势不重复出现),我们设$V(n, \lambda)=(V_{ij})_{4\times 4}$,则静态零曲率方程
\begin{equation}
	S=V^+U - UV = 0
\end{equation}
是包含$16$个未知量的线性方程。 
我们的目的是构造恰当的$V$的形式,使得静态零曲率方程满足下面两个条件:
{\color{blue}
\begin{itemize}
\item (C1): 如果$ \dfrac{\partial U_{ij}}{\partial n}\neq 0$,则$S_{ij}$形如$A+\lambda^k B=0$,其中$A,B$与谱参数$ \lambda$无关,且$U_{ij}$中位势的系数为$\lambda$的$0$次或者$k$次幂。

\item (C2): 如果$ \dfrac{\partial U_{ij}}{\partial n}= 0$,则$S_{ij}$形如$\lambda^k A=0$,其中$A$与谱参数$\lambda$无关。
\end{itemize}
}

接下来我们给出Maple实现的思想和步骤。 下面我们只针对$ \dfrac{\partial U_{ij}}{\partial n}= 0$所对应的$S$的若干个方程(记为$S_0$)进行操作。 我们分为两个步骤: 
{\color{blue}
\begin{itemize}
\item 第一步: 减少未知量的个数。 对于$eq\in S_0$, 如果某个变量可以用其它变量表示出来,则讲$S$中所有该变量替换。 此时,方程数和未知量个数都减少一个。 重复该操作,直到不存在某个变量可以用其它变量表示。 此时$S_0$剩余的式子仍记为$S_0$。

\item 第二步: 平衡$\lambda$。 对于$eq\in S_0$, 如果$\lambda$的最大最小次幂不等, 则为了满足(C2),我们将最低次幂的系数中的未知量替换为$\lambda^k$乘以这些未知量, 使得该式达到平衡。 因此我们需要找到这些未知量,然后对$S$整体进行替换。 重复上述操作。
\end{itemize}
}
一般而言$S_0$满足(C2), 相应的$S-S_0$就满足(C1)。 对于不能满足(C2)或者满足(C2)但不满足(C1)的问题,我们无法给出$V$的形式。

下面我们对程序中的函数做一些说明。
\begin{itemize}
\item {\color{blue}size:} 返回向量,集合或者列表的长度。

\item {\color{blue}format-szce}: 消去(C2)中的$\lambda^k$。

\item {\color{blue}cancel-var}: 返回某个未知量用其它未知量表示的表达式。

\item {\color{blue}reduce-szce}: 将上面的结果代入$S$,减少方程个数。

\item {\color{blue}find-V}: 返回需要乘以$\lambda^k$的未知量。

\item {\color{blue}balance-lambda}: 将上面结果代入$S$, 平衡$\lambda$。

\item {\color{blue}check}: 检查$S$是否满足(C1)和(C2)。

\end{itemize}

基于此, 我们也可以随机生成$U$,看是否可以找到满足(C1)和(C2)的$V$。 我们也用程序实现了这一想法, 这里我们不再描述。 这一方法很容易推广到连续谱问题,这里也不在给出。

我们将程序放在\href{https://github.com/jiandandaoxingfu/derive-hierarchy-V}{Github代码库}, 这里不在附上。

\end{document}